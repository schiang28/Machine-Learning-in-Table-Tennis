\documentclass[journal]{IEEEtai}

\usepackage{xcolor}
\usepackage{nicefrac}
\usepackage{xspace}
\usepackage{graphicx}
\usepackage{amsmath}
\usepackage{amsfonts}
\usepackage[utf8]{inputenc}
\usepackage{enumitem}
\usepackage{nameref}

\usepackage[pagebackref,breaklinks,colorlinks,urlcolor=blue,linkcolor=blue,citecolor=blue]{hyperref}
\usepackage[capitalize]{cleveref}
\usepackage{graphicx}
\usepackage{multirow}
\usepackage{float}
\usepackage{acronym}
\usepackage{epstopdf}
\usepackage{comment}
\usepackage{afterpage}
\usepackage{orcidlink}
\usepackage{verbatim} 

\graphicspath{{./plots/}}
\newcommand{\figref}[1]{Figure~\ref{fig:#1}}
%\newcommand{\eqref}[1]{Equation~\ref{eq:#1}}
\newcommand{\secref}[1]{Section~\ref{sec:#1}}
\newcommand{\algoref}[1]{Algorithm~\ref{algo:#1}}
\newcommand{\chapterref}[1]{Chapter~\ref{chapter:#1}}
\newcommand{\appref}[1]{Appendix~\ref{app:#1}}
\newcommand{\tableref}[1]{Table~\ref{tab:#1}}
\newcommand{\etal}{et al.\xspace}

\DeclareMathOperator*{\argmax}{argmax}



\iftrue

\newcommand{\todo}[1]{\textcolor{red}{\textbf{todo: #1}}}
\newcommand{\denes}[1]{\textcolor{blue}{\textnormal{(DrDenes) #1}}}
\newcommand{\sophie}[1]{\textcolor{orange}{\textnormal{(Sophie) #1}}}

\else
% for final submission
\newcommand{\todo}[1]{}
\fi

\begin{document}

%% \jvol{XX}
%% \jnum{XX}
%% \paper{1234567}
%% \pubyear{2020}
%% \publisheddate{xxxx 00, 0000}
%% \currentdate{xxxx 00, 0000}
%% \doiinfo{TQE.2020.Doi Number}
\title{Supervised Learning for Table Tennis Match Prediction}
\author{\IEEEauthorblockN{Yu-hung Sophie Chiang}, and
    Gyorgy Denes\,\orcidlink{0000-0003-4792-9100}

\thanks{Manuscript submitted for review on  June 2nd 2022} 
\thanks{Yu-hung Sophie Chiang and Gyorgy Denes are with the Perse School, Hills Rd, Cambridge CB2 8QF, UK (correspondence email: gdenes@perse.co.uk)}
}

\markboth{Journal of IEEE Transactions on Artificial Intelligence, Vol. 00, No. 0, Month 2022}
{Chiang and Denes: Supervised Learning for Table Tennis Match Prediction}



\maketitle

%\thispagestyle{plain}
%\pagestyle{plain}

\begin{abstract}
Machine learning, classification and prediction models have applications across a range of fields. Sport analytics is one such increasingly popular field, but most existing work focuses on automated refereeing and injury prevention in mainstream sports. 
Research on other sports, such as table tennis, have only recently started gaining more traction. In this paper, we aim to predict the outcome of table tennis matches; we valuate a range of existing supervised machine learning models to find a reliable predictor, trained on historical player and  match statistics. We also derive 12 features and demonstrate their utility in an ablation study. We found that  logistic regression performed best on the test data, although differences between models were not always significant.  Our results match the accuracy of state-of-the-art in comparable sports, such as tennis.
\end{abstract}

\begin{IEEEImpStatement}
The use of supervised learning models has been proposed for outcome prediction in a number of sports; however, there has been little research done for table tennis. This paper extends the application field in this direction, rigorously evaluating existing models. Furthermore, we also present a set of additional features that increase model accuracy, as validated in the ablation study. The results can serve as a baseline for future table tennis prediction models, and can feed back to prediction research in similar ball sports.
\end{IEEEImpStatement}


\if{false}  % Feel free to add here, but removing from manuscript
If not IEEE TAI, we could consider:
\begin{itemize}
    \item{Computers \& Mathematics with Applications (Elsevier)}
    \item{\sophie{MLSA 2022 : Machine Learning and Data Mining for Sports Analytics}}
\end{itemize}
\fi

% GXD: roughly the structure I was thinking, but feel free to tweak/move bits
\section{Introduction}
Table tennis is one of the quickest and most technical sports, requiring players to respond within milliseconds to a perceived incoming ball trajectory. The final outcome of a game can be influenced by subtle factors, which might be difficult for a human to recognise. 
%The possible inaccuracy in human decision making has thus lead to the application of machine learning techniques to sport result prediction.

Machine learning methods have been frequently used in other sports, such as tennis and football. Proposed applications involve improved training efficiency 
%\denes{consider no citation, and repeat ref later in the text}
as well as result prediction. Specifically, result prediction is of special interest to sport fans, but little has been done in table tennis prediction.
%, the main reason being is the lack of available match data.  GXD: this is an important point, but here it almost implies that this paper adds dat

Table tennis is very fast-paced, with intense rallies having ball speeds of 60-70mph and a rotational speed of 9000rpm. The proximity between players is a lot lower compared to other sports such as tennis and badminton, thus demanding players to have very high reflexes.
There is a variety of in-game data, some which has been analysed in previous studies e.g. Wang \etal \cite{wang2019tac} where data was collected manually, however due to the size and speed of the ball, it is difficult to determine how accurate this is.

Recent developments in  multi-class event spotting and small object tracking has meant that automated data collection from table tennis matches is now attainable. 
In this paper, we propose using some of this freshly available data to train and evaluate  state-of-the-art classification algorithms on both men and women's professional singles matches. %Player statistics collected from historical matches are predominantly used in predicting the outcome, with newly derived information calculated from combining player statistics.

The main contributions of this paper are:
\begin{itemize}
    \item a quantitative evaluation of different ML models for table tennis match prediction
    \item  an investigation on engineering new features from the raw data
\end{itemize}

%The main contributions of this paper focuses on bridging the gap between data collection and predictive analysis in table tennis, discussing relevant state-of-the-art machine learning models and reporting the model with the highest predictive performance using rigorous quantitative evaluation. 

The rest of the paper is organised as follows: First we review the relevant publications on sports prediction and we describe the OSAI dataset on which our work is built on. Then, we describe the proposed feature set. Finally, we evaluate the performance of a range of state-of-the-art models with ablation studies and hyperparameter tuning.
\section{Background}
%\subsection{Table tennis}
%\denes{not sure about section headers, but it would make sense to separate out the background from the ML. Also, can we have citations for the data, please?}
Table tennis is played competitively across the world. A table tennis match consists of a sequence of sets; in a professional singles match, the first player to win best of seven sets wins the match. In doubles (two teams of two players play against each other), the first team to win best of five sets wins the match. This paper will be looking at modelling professional singles matches only.
In a set, the first player that earns at least eleven points and at least two more than their opponent wins the set. Each player serves twice before alternating, however, if the score reaches at least 10-10, each player serves only once before alternating.
The sport has proven to be very popular, with more than three hundred million players worldwide. %, a large proportion residing in East Asia.
The full set of rules are published by the International Table Tennis Federation \cite{ITTF}.

\section{Related Work} \label{sec:relatedwork}
\subsection{Machine Learning}
%\denes{We might want to just remove this subsection; it might be worth taking a look at some similar publications to check whether they even bother to include supervised learning in their related work (I suspect not)}.
Machine learning (ML) is a branch of artificial intelligence that has been successfully applied to many areas of industry and science, including disease diagnosis in medicine \cite{kourou2015machine}, pattern recognition \cite{weiss1989empirical}, computer vision \cite{khan2020machine} and bioinformatics \cite{larranaga2006machine}.
%This paper will be using supervised ML methods, as match instances are labelled and used to estimate the desired outcome.
The problem of predicting a table tennis match can be thought of as a supervised binary classification problem, with ground truth match outcome labels widely available. %as data is categorized into one of two possible classes.


\subsection{ML in Sports}
%\todo{A general overview of ML in sports}

In the past, manual data collection methods for sports have typically proven time consuming and prone to human error and bias. Recent improvements in data capture has sparked interest in automatic data collection and analysis for a range of sports. Xing \etal \cite{xing2010multiple} proposed a dual-mode two-way Bayesian inference approach to track multiple highly dynamic and interactive players from videos in team sports such as basketball, football and hockey. Claudino \etal \cite{claudino2019current} used different ML methods, such as neural networks and decision tree classifiers, to investigate injury risk and performance in football, basketball, handball and volleyball. Davoodi and Khanteymoori \cite{davoodi2010horse} used neural networks for horse racing prediction, where eight features were used as input nodes to each neural network. This included information such as horse weight and race distance, to predict the eventual finishing time and rank of every horse in a race. 

% quite interesting, but potentially too much about horses
%Some horses lacked performance data, and had to be excluded from the analysis.


%therefore it would not be adequate to use it's history for prediction. As such, these horses had to be removed from the dataset to achieve better results.

Applications of ML in sports can help with players and performance analysts in identifying critical factors that contribute to winning. Appropriate tactics can be identified in maximising player performance.
Aside from formulating strategies to win matches, using machine learning methods for sport result prediction has become popular due the expanding domain in betting \todo{[cit.]}, which necessitates high predictive accuracy. Other applications include automated scouting and recruitment \cite{bunker2019machine} and umpiring assistance \cite{vzemgulys2018recognition}. \denes{Again, if we coul baseline against some betting odds, that would be a fun evaluation}


\subsection{Prediction in Tennis}
For tennis, Knottenbelt \etal \cite{knottenbelt2012common} proposed a common opponent model to find a pre-play estimate of the probability of a player winning a professional singles \textit{tennis} match. This was achieved by analysing match statistics for opponents that both players encountered in the past.%, which provided for a fair basis comparison. 
The model computed the probability of each player winning a point on their serve, and hence the match. The authors claimed an approximate return of investment of 6.85\% when put into the betting market for over four major tennis tournaments in 2011. \denes{What was their accuracy?} \denes{would it be easy to adapt and eval their common opponent model fortable tennis? That would be a cool base line as well}

Barnett and Clarke \cite{barnett2005combining} use historical data from past matches to predict the probability of a player winning a single point. Clarke and Dyte \cite{clarke2000using} use a year's worth of tournament results to predict the outcome of a match using player rating points. Such an approach which maps player ability to a single rank can fail to capture more complex factors, such as  a player's susceptibility to a certain skill or strategy. In this paper we demonstrate the benefits of using more complex features in the similar sport of table tennis.

%Due to similarities between tennis and table tennis, certain concepts can be applied from works that have been conducted in tennis.

%Both are ideal sports to apply hierarchical probability models to; a table tennis match consists of a sequence of sets, which consists of a sequence of points.

\subsection{ML in Table tennis}
In table tennis, ML applications have focused so far on computer vision and automated data collection. % and  a number of computer vision based approaches have been applied. 
Voeikov \etal \cite{voeikov2020ttnet} proposed a neural network (TTNet) that allowed for real-time processing of high-resolution table tennis videos. This can extract temporal and spatial data, such as ball detection and in-game events, and is potentially capable of substituting manual data collection by sport scouts, in addition to assisting with referee decision making. Zhang \etal \cite{zhang2010visual} are able to compute the 3D coordinates of a table tennis ball by its image coordinates. This allows the trajectory, the landing and striking point to be calculated.

We build on these existing works by utilising data by Voeikov \etal \cite{voeikov2020ttnet}, and apply it to the yet unexplored problem of supervised table tennis match prediction.
%, however, this study does not extend to predicting the match outcome.

%Current works in table tennis mainly focus on ball detection or calculating ball speed, where the focus has been on data collection rather than analysis. Predicting the result of a match has not yet been addressed, thus leading to the development of this project.



\begin{figure}[ht]
\centering

\includegraphics[width=8.5cm]{plots/tablesequence.pdf}
\caption{Progression of a rally demonstrating the landing point of each ball bounce. Yellow indicates service which starts a rally and red indicates an error ending the rally. Green indicates all other ball bounces \cite{OSAI}.}
%\denes{Please remove all margins on these plots}}

\label{fig:sequence}
\end{figure}

\begin{figure}[ht]
\centering

%\vspace{-2em}
\includegraphics[width=8cm]{plots/tableheatmaplot.pdf}
\caption{Location of the last bounce of the winning ball, summed over an example match. Each side of the table is split into nine equal parts. \cite{OSAI}.}

\label{fig:pos}
\end{figure}


\begin{figure}[t]
\centering
\includegraphics[width=8cm]{plots/forehandvsbackhand.pdf}
\includegraphics[width=8cm]{plots/shortvslongrally.pdf}
\caption{Number of points won by forehand vs. backhand (top);  by a short vs. long rally (bottom) in an example match \cite{OSAI}.}

\label{fig:svlr}
\end{figure}

\section{Dataset} \label{sec:dataset}
Our primary source of data are automatic captures from TTNet \cite{voeikov2020ttnet}, released by OSAI \cite{OSAI}. We use Tokyo 2020 Olympics and Tischtennis-Bundesliga (German table tennis league) data, which include men and women's singles matches. Potential features include player  rank and in-match statistics such as percentage of points won on serve and receive, stroke  and error types. Match progression can be plotted for each set, recording the location of each ball bounce (see \figref{sequence}).

%Interactive maps that demonstrated the ball position of each shot on the table, as well as the stroke type were also accessible. The progression of a rally and location of each ball bounce can be mapped into a sequence (see \figref{sequence}).

To reduce the dimensionality of the problem, a rally can be represented as the location of the winning shot. Furthermore, each half of the table can be split into nine equal sections, and the location of winning shots can be grouped (\figref{pos}). Further grouping can involve the number of forehands and the number of backhands used to win a point, or whether it was a `short' or `long' rally (\figref{svlr}). % The use of these features is discussed in  \secref{features}.
%\denes{which version? Fig 3 or 4?}.
Samples with missing data entries were removed from the dataset.

 
 %One of the main challenges in constructing a successful result predictor is the selection of salient features. To address this, we carefully hand picked features that we thought would be the most influential in a match based on existing domain knowledge on the problem.
\section{Feature Engineering \& Selection} \label{features}
Rich data is valuable, however, irrelevant or redundant variables can increase both training and inference time, as well as decrease a model's performance, therefore choosing appropriate features to input into a classification model is very important.

\subsection{Match Representation}
In supervised machine learning, a set of labelled data is required for the model to train on. In the context of table tennis prediction, each match corresponded to two instances of data, one from the perspective of each player respectively, where every sample is composed of two elements:
\begin{itemize}
    \item A vector of input features ($\underline{x}$) consisting of player and match statistics
    \item The target variable ($y$), indicating the result of the match that corresponds to its respective sample
\end{itemize}
The outcome of the match for player $i$ is defined as follows:

$$
    y =
    \begin{cases}
    1, &\text{if $player_i$ wins} \\ 
    -1, &\text{if $player_i$ loses} \\
    \end{cases}
$$
As any incomplete matches were removed from the dataset, combined with the inability to draw in table tennis, there is no other possible outcome.

\subsection{Feature Engineering}
In addition to the features extracted from the dataset in Section \ref{dataset}, we combined players' statistics to form new features \cite{barnett2005combining}. Using pre-existing knowledge on the sport, adding combinations of player statistics as features may improve the predictive model. These features were calculated as differences between different player statistics as this considers the characteristics of \textit{both} players participating in a match. This was inspired by Sipko and Knottenbelt \cite{sipko2015machine} and Cornman \etal \cite{cornman2017machine}, who both use features calculated as differences to predict the outcome of a tennis match.

In table tennis, there are only two possible outcomes of a match; a win or a loss. Unlike team sports, a combination of players with each of varying individual ability and skill level do not need to be considered in the predictive outcome. Due to this, the likelihood of player substitutions nor offensive and defensive combinations do not need to be analysed.

For the final feature set with abbreviated feature names and explanations, see Table \ref{table:nonlin}. Newly derived features are indicated with * and more detailed explanations can be found in subsections \ref{rankdiff}, \ref{advantage} and \ref{balance}. For the purposes of this paper, a long rally is defined as a rally of at least five shots, and a short rally is any rally of length under five.

\begin{table}[ht]
\caption{Feature Summary} % title of Table
\centering % used for centering table
\setlength{\tabcolsep}{3pt}
\scalebox{1.2}{%
\begin{tabular}{c c} % centered columns (4 columns)
\hline\hline %inserts double horizontal lines
Feature & Explanation \\ [0.5ex] % inserts table

%heading
\hline % inserts single horizontal line
SP & percentage of total points won on serve \\
RP & percentage of total points won on receive \\
LRP & percentage of total points won on a long rally \\
SRP & percentage of total points won on a short rally \\
FHP & percentage of total points won on a forehand \\
BHP & percentage of total points won on a backhand \\ 
RANK & player ranking \\
RANKDIFF* & difference in rank between opponents \\
SA* & player serve advantage \\
SRA* & player short rally advantage \\
FHA* & player forehand advantage \\
BALANCE* & measure of how well rounded a player is \\
[1ex] % [1ex] adds vertical space

\hline %inserts single line
\end{tabular}}
\label{table:nonlin} % is used to refer this table in the text
\end{table}

\subsubsection{Rank Difference} \label{rankdiff}

The feature \textit{RANKDIFF} was constructed by calculating the difference between rankings of two opponents.
$$
RANKDIFF = \begin{cases}
RANK_i - RANK_j &\text{for player $i$} \\
RANK_j - RANK_i &\text{for player $j$} \\
\end{cases}
$$
where $RANK_i$ and $RANK_j$ are the rankings of players $i$ and $j$ respectively at the time of the match. Therefore if a player's rank is better (i.e. lower numerical value) than their opponent's rank, $RANKDIFF$ will be a negative value, and vice versa. However, for some match instances where the ranking of both players are above 100, the feature $RANKDIFF$ is considered to be 0. This is due to the fact that the lower the rank of a player, the more likely it is that there will be other players of a similar standard where rank doesn't accurately represent the standard of a player.

For example, players of rank 2 and rank 7 is much more likely to have an accurate depiction of their standard in comparison to two players of rank 150 and 155, despite the rank difference being the same. Therefore, if both players have a rank of below 100, the expected difference in skill level is considered to have no benefit.

\subsubsection{Serve Advantage} \label{advantage}
The serve advantage of a player is calculated as the difference between their serve and receive winning percentage. This depicts the contrast as to how likely a player is to win a point if they are serving, compared to if they are on receive. Subsequently, the advantage a respective player has in a short rally over a long rally, as well as the advantage a respective player has in a forehand stroke over a backhand stroke, can be calculated therefrom.

\subsubsection{Balance} \label{balance}
An attempt to measure the \textit{completeness} of a player can be calculated by taking the average of serve, short rally and forehand advantage:
$$
BALANCE = \frac{|SA|+|SRA|+|FHA|}{3}
$$
Players of a higher skill level tend to have fewer weaknesses and are stronger in more aspects of the game, therefore the feature $BALANCE$ indicates the overall well-roundness of a player's ability. \label{engineer}

\subsection{Feature Scaling}
Different features tend to have a varying range of values, therefore it is best practice to scale features as part of data pre-processing prior to learning. \textit{Standardization} is a scaling technique to centre values around the mean with a unit standard deviation \cite{bollegala2017dynamic}, and was performed for each feature. 

\section{Models} \label{models}
This paper uses Scikit-Learn's implementation to apply different models to the dataset \cite{pedregosa2011scikit}.

\subsection{Logistic Regression}
The logistic function $\sigma(t)$ is defined as follows:
$$
\sigma(x) = \frac{1}{1+e^{-x}}
$$

The logistic function maps any input value $x$ where $x = \{x\in \mathbb{R} \}$ to a value between 0 and 1, allowing the output to be interpreted as a probability. If this probability is considered to be over 0.5, it can be classified as true, where the player is predicted to win the match, otherwise the result is classified as false. The model gives the best reproduction of match outcomes for the training set by minimising the \textit{logistic loss} function:
$$
L(p) = -\frac{1}{n} \sum_{i=1}^n p_i \log(y_i) + (1-p_i)\log(1-y_i)
$$
\begin{equation*}
\begin{gathered}
    n = \text{number of matches} \\
    p_i = \text{predicted probability of a player winning match $i$} \\
    y_i = \text{outcome of match $i$},
\end{gathered}
\end{equation*}
where a loss function measures the disparity between observations and their estimated fits \cite{hazan2014logistic}.

\subsection{Random Forest}
A random forest is a classifier consisting of a collection of simpler tree-structured classifiers $\{h(\textbf{x},\theta_k),\ k=1,...\}$, where the $\{\theta_k\}$ are independent identically distributed random vectors and each tree casts a unit vote for the most popular class at input $\textbf{x}$. For the $k$th tree, a random vector $\theta_k$ is generated and a tree is grown using $\theta_k$ and the training set, to produce a classifier $h(\textbf{x}, \theta_k)$ \cite{breiman2001random}. After a large number of trees are produced, the output of a random forest model is the class that receives the most votes. Decision trees tend to be simple to interpret and ``quick'' to train, making it a popular ML technique.

\subsection{Support Vector Machines (SVM)}
SVMs have been used in predicting tennis match outcome \cite{cornman2017machine}. The idea is that SVMs map an input vector in a feature space of $n$ dimensions, where $n$ is the number of features. The optimal hyperplane is identified which separates data points into two classes. This is known as the decision boundary, and the marginal distance between this boundary and the instances closest to the boundary is maximized. The existence of a decision boundary can allow for any detection of miss-classification. SVM algorithms use a set of mathematical functions that are defined as \textit{kernels}, and different SVM algorithms use different type of kernel functions. Different kernels including linear, polynomial, sigmoid and radial basis function (RBF) were used for the purposes of this study.

\subsection{Multilayer Perceptron Neural Networks (MLP)}
An MLP neural network is a mathematical model inspired by human neurons that consists of one or more hidden layers in-between it's input and output layers, and is designed to imitate the behaviour of biological neurons in the human brain. The network consists of mutually connected artificial neurons, where neurons are organised in layers, and connections are directed from lower layers to upper layers. Neurons from the same layer are not interconnected \cite{noriega2005multilayer}.

Each connection between two neurons has an associated weight, and in the process of learning and training an MLP model, these weights are adjusted such that there is a minimal difference between the model output and the desired output. 
\section{Models} \label{sec:models}
\begin{figure*}[ht]

\centering

\includegraphics[width=18cm]{plots/confusionmatricesr1.pdf}


\label{confusionmatricesr1}
\end{figure*}
\begin{figure*}[ht]
\vspace{-2em}
\centering
\includegraphics[width=14cm]{plots/confusionmatricesr2.pdf}

\caption{Confusion matrices comparing the predicted and actual outcomes of test cases for each trained model.}
%\denes{Please reduce outside margins and consider removing the legend}}

\label{fig:confusionmatrices}
\centering
\end{figure*}

\denes{This is where we can still cut out quit a lot if we want to}
This paper uses Scikit-Learn's implementation to apply different models to the dataset \cite{pedregosa2011scikit}.

\subsection{Logistic Regression}
The logistic function $\sigma(t)$ is defined as follows:
\begin{equation}
    \sigma(x) = \frac{1}{1+e^{-x}}
\end{equation}

The logistic function maps any input value $x$ where $x = \{x\in \mathbb{R} \}$ to a value between 0 and 1, allowing the output to be interpreted as a probability. If this probability is considered to be over 0.5, it can be classified as true, where the player is predicted to win the match, otherwise the result is classified as false. The model gives the best reproduction of match outcomes for the training set by minimising the \textit{logistic loss} function:

\begin{equation}
    (p) = -\frac{1}{n} \sum_{i=1}^n p_i \log(y_i) + (1-p_i)\log(1-y_i)
\end{equation}

$$
\begin{gathered}
    n = \text{number of matches} \\
    p_i = \text{predicted probability of a player winning match $i$} \\
    y_i = \text{outcome of match $i$},
\end{gathered}
$$

where a loss function measures the disparity between observations and their estimated fits \cite{hazan2014logistic}.

\subsection{Random Forest}
A random forest is a classifier consisting of a collection of simpler tree-structured classifiers $\{h(\textbf{x},\theta_k),\ k=1,...\}$, where the $\{\theta_k\}$ are independent identically distributed random vectors and each tree casts a unit vote for the most popular class at input $\textbf{x}$. For the $k$th tree, a random vector $\theta_k$ is generated and a tree is grown using $\theta_k$ and the training set, to produce a classifier $h(\textbf{x}, \theta_k)$ \cite{breiman2001random}. After a large number of trees are produced, the output of a random forest model is the class that receives the most votes. Decision trees tend to be simple to interpret and ``quick'' to train, making it a popular ML technique.

\subsection{Support Vector Machines (SVM)}
SVMs have been used in predicting tennis match outcome \cite{cornman2017machine}. The idea is that SVMs map an input vector in a feature space of $n$ dimensions, where $n$ is the number of features. The optimal hyperplane is identified which separates data points into two classes. This is known as the decision boundary, and the marginal distance between this boundary and the instances closest to the boundary is maximized. The existence of a decision boundary can allow for any detection of miss-classification. SVM algorithms use a set of mathematical functions that are defined as \textit{kernels}, and different SVM algorithms use different type of kernel functions. Different kernels including linear, polynomial, sigmoid and radial basis function (RBF) were used for the purposes of this study.

\subsection{Multilayer Perceptron Neural Networks (MLP)}
An MLP neural network is a mathematical model inspired by human neurons that consists of one or more hidden layers in-between it's input and output layers, and is designed to imitate the behaviour of biological neurons in the human brain. The network consists of mutually connected artificial neurons, where neurons are organised in layers, and connections are directed from lower layers to upper layers. Neurons from the same layer are not interconnected \cite{noriega2005multilayer}.

Each connection between two neurons has an associated weight, and in the process of learning and training an MLP model, these weights are adjusted such that there is a minimal difference between the model output and the desired output. 

\subsection{Evaluating Models} \label{evalmodels}
%\todo{shorten this}
To compare the performance of different model predictions, we calculated the \textit{accuracy} of each model
\begin{equation}
    \text{accuracy} = \frac{tp+tn}{tp+tn+fp+fn},
\end{equation}
where $tp$ and $tn$ are true positives and true negatives, and $fp$ and $fn$ are false positives and false negatives respectively.  %\cite{vanwinckelen2012estimating}. 
%\sophie{should figures 2 and 3 be kept?}
%\denes{No, but we could report all the confusion matrices for each model (possibly in the appendices).}
To get a more balanced idea about model performance, we also compute F1 scores as:
%Precision is defined as the proportion of true positives to the total number of predictions predicted positive, and is a percentage of returned results which are relevant. Recall is defined to be the proportion of true positives to the total number of actual positives, and is a percentage of relevant data which have been correctly classified \cite{buckland1994relationship}.
%\begin{equation}
\begin{gather}
    \text{precision} = \frac{tp}{tp+fp} \quad
 \text{recall} = \frac{tp}{tp+fn}\\
 \vspace{0.0cm} \notag \\
F1 = \frac{2\times\text{precision} \times \text{recall}}{\text{precision} + \text{recall}},
\end{gather}
%\end{equation}
which is effectively an F measure with $\beta=1$ \cite{sokolova2006beyond}.
%Both metrics are important to take into consideration, and ultimately we use the F1 measure, which can be interpreted as the harmonic mean between precision and recall \cite{buckland1994relationship}, and accuracy score for evaluation of each classifier.

% An issue associated with training models is the possibility for the model to \textit{overfit}.
To avoid over-fitting, we used 5-fold cross validation. The  dataset was split in training:validation:test in a 72:18:10 ratio. 10\% of the original dataset was kept as a test set to validate hyperparameter tuning. The remaining 90\% of data was split in an 80:20 ratio for the 5-fold training; 80\% to train the model, 20\% to optimize hyperparameters. %The model trains on the training set, and the validation set is used for optimising the hyperparameters of the model.
%Overfitting occurs when a model captures unwanted bias and noise in the data that it negatively impacts the performance of a model. The model corresponds to it's initial training data too well, and fails to predict unseen data reliably. In order to limit overfitting, a popular re-sampling technique called $k$-fold cross validation is used.
%In cross validation, the dataset is split into $k$ random subsets, known as folds, and one is selected as a test set for the model to test on, while the others are used as a training set for the model to train on. This is repeated $k$ times where a different subset of data is used as the test set each time, and the overall performance of the model is calculated as the average of accuracy scores for each iteration \cite{berrar2019cross}.

\subsection{Hyperparameter Tuning} \label{hpt}
We used a brute-force grid search to fine tune parameters of the model that are outside the usual training domain (hyperparameters) e.g. the number of trees in a random forest classifier.
%For each model, the \textit{hyperparameters}, parameters that are not optimised by the training algorithm e.g. the number of trees in a random forest classifier, were tuned manually by using a grid search to test different values.
%A grid search uses brute force to search the entire space for different hyperparameter configurations.
The best combination of hyperparameters for a model is determined by whichever has the highest accuracy on the validation set using 5-fold cross validation.

For logistic regression, the type of solver, penalty function and the $C$  terms value were adjusted. $C$ is a regularisation term; 
%\textit{Regularisation} prevents overfitting of training data by penalising large weights when training a logistic regression predictor, and the parameter $C$ mentioned is used to control the effect of this\cite{ahmadian1998regularisation}. 
the lower the value of $C$, the stronger the effect of regularisation. We found that  $C=1.0$, and `liblinear' solver resulted in the best average accuracy. In terms of regularisation, L2 regularisation gave better results than L1 (Fig. \ref{fig:learningcurve}).

%\denes{Could we plot l1 vs l2 here? That would be quite convincing}.

For SVMs, the two main hyperparameters that were adjusted were the kernel type and penalty value $C$. Using a linear kernel and $C=0.2$ gave the highest F1 score on the test set compared to other kernels. The learning curve for an SVM model using a linear kernel is shown in Fig. \ref{fig:learningcurve}.

\begin{figure}[H]
\includegraphics[width=8.cm]{plots/chiang5.pdf}
\caption{Logistic regression and linear kernel SVM learning curves. The difference in F1 score for L1 and L2 regularisation is also illustrated.}
\label{fig:learningcurve}
\centering
\end{figure}

\section{Experimental Results} \label{experresults}
Our main results are reported in Table~\ref{results} and \figref{confusionmatrices}.
%The test set is completely left out from the training process of the model, and is only used during evaluation. This is to see how well models generalise to unseen data which replicate new match data, as the test set is never used before evaluation. During evaluation, all other data is used for training the model (training and validation).
Both accuracy and F1 score are reported for the validation and test sets. The standard error for each score for the validation set is reported as a basis of defining uncertainty. The validation column shows that most models perform comparably with approx. 70\% accuracy. This value is also comparable to state-of-the-art metrics in \textit{tennis} match prediction.

F1 scores indicate that MLP Neural Networks (with a \textit{relu} activation) slightly over-perform their competitors, but the difference is not significant. The hidden layer size was set to 2 and the maximum number of iterations the solver iterates was chosen to be 200. The solver for weight optimization is set to `lbfgs', a quasi-Newton optimizer. The learning rate for scheduling weight updates is set to constant. However, the generic layered structure of a neural network has proven to be time consuming. Additionally, this technique is considered a `black box' technology, and finding out why a neural network has outstanding or even poor performance is difficult \cite{noriega2005multilayer}.

%On the test set, logistic regression seems to perform  noticeably better than other models. One interesting observation is that this is the only classifier that noticeably benefited the most from hyperparameter tuning. This might be partially due to the fact that we report the different kernels of SVMs separately.

\begin{table}[t]
\caption{Model performance comparing validation and test sets}
\label{results}
\centering
\setlength{\tabcolsep}{3pt}
\scalebox{1.05}{%
\begin{tabular}{ l|c c|c c }

\multirow{2}{4em}{Model} &
\multicolumn{2}{|c|}{Validation set} &
\multicolumn{2}{|c}{Test set} \\
\cline{2-5}
  & Acc & F1 & Acc & F1 \\

\hline \hline 
Logistic Regression & 0.699$\pm{0.024}$ & 0.705$\pm{0.023}$ & 0.722 & 0.706 \\
Random Forest & 0.677$\pm{0.032}$ & 0.688$\pm{0.033}$ & 0.667 & 0.684 \\
Support Vector Machine & & \\
$\rightarrow$ Linear & 0.696$\pm{0.029}$ & 0.690$\pm{0.035}$ & 0.639 & 0.629 \\
$\rightarrow$ RBF    & 0.700$\pm{0.025}$ & 0.677$\pm{0.034}$ & 0.667 & 0.600 \\
$\rightarrow$ Polynomial & 0.705$\pm{0.021}$ & 0.685$\pm{0.021}$ & 0.611 & 0.563\\
$\rightarrow$ Sigmoid & 0.705$\pm{0.017}$ & 0.690$\pm{0.019}$ & 0.694 & 0.621 \\
MLP Neural Network & 0.696$\pm{0.019}$ & 0.708$\pm{0.020}$ & 0.694 & 0.703\\
[1ex]
\hline
\end{tabular}}
\end{table}


\begin{figure}[ht]
\centering
\includegraphics[width=8.cm]{plots/chiang7.pdf}
\caption{ROC Learning Curves for the overall performance of each model.}
%\denes{As discussed, please replot without axis / bigger fonts. Or actually, why not do one single ROC plot (you have 4 colours anyway?}}

\label{fig:roc}
\end{figure}


Receiving operating characteristics (ROC, \figref{roc}) support the quantitative results. The kernel choice for SVM models makes a noticeable difference; the areas under the ROC curves are otherwise comparable for all other models.
%\figref{par} shows that the precision vs. recall trade-off is once again comparable for all models, except for the random forest classifier. While this seems like an interesting phenomenon, 

%ROC curves demonstrate the performance of a classification model for all classification thresholds. The closer the apex of the curve is to the upper left hand corner, the greater the model's discriminatory ability \cite{fan2006understanding}. This is plotted for each type of classifier, including the SVM model using a linear kernel. 
%For each trained model, we use it's accuracy score, F1 measure and the area under its ROC curve to evaluate overall performance\denes{do we report area as well anywhere?}.

%\section{Discussion} \label{discussion}
%The model that achieved the highest F1 score on the validation set were MLP neural networks. The difference in accuracy and F1 score between the validation and test set was shown to be smaller in comparison to SVMs and random forest.

One qualitative advantage of using a random forest classifier is its training speed, which made hyperparameter tuning easier. The maximum number of levels in each decision tree was set to 80, the maximum number of features considered for splitting a node was set to 4, the minimum number of data points allowed in a leaf node was set to 4 and the number of trees that were in the classifier was set to 200.

\begin{figure}[ht]

\includegraphics[width=8.5cm]{plots/feature_importance.pdf}
\caption{Importance of features from random forest classifier based on Gini impurity. In our dataset, RANKDIFF appears to be the most important feature.}
%\denes{Can we put * on the derived features, please?}. }
\vspace{-1em}
\label{fig:fig4}
\centering
\end{figure}

Another significant advantage of a random forest classifier is that the importance of features can also be extracted and visualised.  \figref{fig4} shows this as the the mean decrease in Gini impurity for features across all trees. The impurity of a node is the probability of a specific feature being classified incorrectly assuming that it is selected randomly \cite{cassidy2014calculating}.

\subsection{Ablation Study}
\figref{fig4} predicts that the most important feature in a random forest model is  RANKDIFF, which justifies the inclusion of hand-crafted features. To reinforce this finding, we report the accuracy and F1 score for each model with and without the derived features (see \secref{engineer}). All scores are lower for models that don't use newly derived features, and accuracy score is significantly lower in SVMs compared to other models. Detailed results  are in Table \ref{results2}.

\begin{table}[ht]
\caption{Model performance with and without newly derived features}
\label{results2}
\centering
\setlength{\tabcolsep}{8pt}
\scalebox{1.05}{%
\begin{tabular}{ l|c c|c c }

\multirow{2}{4em}{Model} &
\multicolumn{2}{|c|}{With} &
\multicolumn{2}{|c}{Without} \\
\cline{2-5}
  & Acc & F1 & Acc & F1 \\

\hline \hline 
Logistic Regression & 0.699 & 0.705 & 0.631 & 0.668 \\
Random Forest & 0.677 & 0.688 & 0.661 & 0.673 \\
Support Vector Machine & & \\
$\rightarrow$ Linear & 0.696 & 0.690 & 0.556 & 0.619 \\
$\rightarrow$ RBF    & 0.700 & 0.677 & 0.500 & 0.591 \\
$\rightarrow$ Polynomial & 0.705 & 0.685 & 0.500 & 0.640\\
$\rightarrow$ Sigmoid & 0.705 & 0.690 & 0.472 & 0.642 \\
MLP Neural Network & 0.696 & 0.708 & 0.639 & 0.683\\
[1ex]
\hline
\end{tabular}}
\end{table}

\section{Conclusion} \label{conc}
In this paper, the concepts of machine learning were discussed and how different supervised ML methods and classification algorithms could be used in table tennis match prediction. 
This paper follows a state-of-the art approach in training and evaluating different machine learning models. We explain the theory behind each model that is used, and how it's performance was evaluated.
The original dataset retrieved from OSAI \cite{OSAI}, in addition to deriving new features were used, and the difference in performance of models with and without newly derived features were compared.
When taking into consideration accuracy and F1 score on the validation and test set, combined with the area under it's ROC curve, the model which achieved the highest evaluated score was by using logistic regression.
From Figure \ref{fig4}, the feature which is shown to be the most important is $RANKDIFF$, a measure of the difference between rank in opponents in an attempt to quantify both players' skill difference. 
Future works could focus on a selection of the most important features established from the random forest model.

\section*{Acknowledgements}
The authors would also like to thank the OSAI team for granting permission to use their dataset.

\bibliographystyle{ieeetr}
\bibliography{References}
\end{document}
