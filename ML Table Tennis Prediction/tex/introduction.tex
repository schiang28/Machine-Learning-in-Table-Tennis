\section{Introduction}
Table tennis is a quick and highly technical sport, requiring players to respond to an incoming ball trajectory within milliseconds. Rallies are intense, with ball speeds of 60--70mph and rotational speeds of 9000rpm. The proximity between players is  a lot lower compared to similar sports such as tennis and badminton. The  outcome of a game can be influenced by subtle factors, which can be hard for a human to recognize.
%The possible inaccuracy in human decision making has thus lead to the application of machine learning techniques to sport result prediction.

Machine learning methods have been used frequently  in other sports, such as tennis and football. Proposed applications involve improved training efficiency 
%\denes{consider no citation, and repeat ref later in the text}
as well as result prediction. Specifically, result prediction is of high concern to sport fans, but little has been done in table tennis prediction.
%, the main reason being is the lack of available match data.  GXD: this is an important point, but here it almost implies that this paper adds dat


While manually-collected datasets alongside some analysis have been available in the past \cite{wang2019tac}, it is only recent developments in  multi-class event spotting and small object tracking that made accurate, detailed in-game data attainable.
In this paper, we propose using some of this freshly available data to train and evaluate  state-of-the-art classification algorithms on both men and women's professional singles matches. %Player statistics collected from historical matches are predominantly used in predicting the outcome, with newly derived information calculated from combining player statistics.

% moved to impact statement
%The main contributions of this paper are:
%\begin{itemize}
%    \item a quantitative evaluation of different ML models for table tennis match prediction
%    \item  an investigation on engineering new features from the raw data
%\end{itemize}

%The main contributions of this paper focuses on bridging the gap between data collection and predictive analysis in table tennis, discussing relevant state-of-the-art machine learning models and reporting the model with the highest predictive performance using rigorous quantitative evaluation. 

The paper is organized as follows: we first review relevant publications on sports prediction and describe the OSAI dataset on which our work is built. Then, we describe the proposed feature set. Finally, we evaluate the performance of three state-of-the-art models and perform a feature ablation.