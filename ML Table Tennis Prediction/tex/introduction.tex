\section{Introduction}
Table tennis is one of the quickest and most technical sports, requiring players to respond within milliseconds to a perceived incoming ball trajectory. The final outcome of a game can be influenced by subtle factors, which might be difficult for a human to recognise. 
%The possible inaccuracy in human decision making has thus lead to the application of machine learning techniques to sport result prediction.

Machine learning methods have been frequently used in other sports, such as tennis and football. Proposed applications involve improved training efficiency 
%\denes{consider no citation, and repeat ref later in the text}
as well as result prediction. Specifically, result prediction is of special interest to sport fans, but little has been done in table tennis prediction.
%, the main reason being is the lack of available match data.  GXD: this is an important point, but here it almost implies that this paper adds dat

Table tennis is very fast-paced, with intense rallies having ball speeds of 60-70mph and a rotational speed of 9000rpm. The proximity between players is a lot lower compared to other sports such as tennis and badminton, thus demanding players to have very high reflexes.
There is a variety of in-game data, some which has been analysed in previous studies e.g. Wang \etal \cite{wang2019tac} where data was collected manually, however due to the size and speed of the ball, it is difficult to determine how accurate this is.

Recent developments in  multi-class event spotting and small object tracking has meant that automated data collection from table tennis matches is now attainable. 
In this paper, we propose using some of this freshly available data to train and evaluate  state-of-the-art classification algorithms on both men and women's professional singles matches. %Player statistics collected from historical matches are predominantly used in predicting the outcome, with newly derived information calculated from combining player statistics.

The main contributions of this paper are:
\begin{itemize}
    \item a quantitative evaluation of different ML models for table tennis match prediction
    \item  an investigation on engineering new features from the raw data
\end{itemize}

%The main contributions of this paper focuses on bridging the gap between data collection and predictive analysis in table tennis, discussing relevant state-of-the-art machine learning models and reporting the model with the highest predictive performance using rigorous quantitative evaluation. 

The rest of the paper is organised as follows: First we review the relevant publications on sports prediction and we describe the OSAI dataset on which our work is built on. Then, we describe the proposed feature set. Finally, we evaluate the performance of a range of state-of-the-art models with ablation studies and hyperparameter tuning.