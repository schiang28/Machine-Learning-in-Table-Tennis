\section{Feature Engineering \& Selection} \label{features}
Rich data is valuable, however, irrelevant or redundant variables can increase both training and inference time, as well as decrease a model's performance, therefore choosing appropriate features to input into a classification model is very important.

\subsection{Match Representation}
In supervised machine learning, a set of labelled data is required for the model to train on. In the context of table tennis prediction, each match corresponded to two instances of data, one from the perspective of each player respectively, where every sample is composed of two elements:
\begin{itemize}
    \item A vector of input features ($\underline{x}$) consisting of player and match statistics
    \item The target variable ($y$), indicating the result of the match that corresponds to its respective sample
\end{itemize}
The outcome of the match for player $i$ is defined as follows:

$$
    y =
    \begin{cases}
    1, &\text{if $player_i$ wins} \\ 
    -1, &\text{if $player_i$ loses} \\
    \end{cases}
$$
As any incomplete matches were removed from the dataset, combined with the inability to draw in table tennis, there is no other possible outcome.

\subsection{Feature Engineering}
In addition to the features extracted from the dataset in Section \ref{dataset}, we combined players' statistics to form new features \cite{barnett2005combining}. Using pre-existing knowledge on the sport, adding combinations of player statistics as features may improve the predictive model. These features were calculated as differences between different player statistics as this considers the characteristics of \textit{both} players participating in a match. This was inspired by Sipko and Knottenbelt \cite{sipko2015machine} and Cornman \etal \cite{cornman2017machine}, who both use features calculated as differences to predict the outcome of a tennis match.

In table tennis, there are only two possible outcomes of a match; a win or a loss. Unlike team sports, a combination of players with each of varying individual ability and skill level do not need to be considered in the predictive outcome. Due to this, the likelihood of player substitutions nor offensive and defensive combinations do not need to be analysed.

For the final feature set with abbreviated feature names and explanations, see Table \ref{table:nonlin}. Newly derived features are indicated with * and more detailed explanations can be found in subsections \ref{sp} to \ref{balance}. For the purposes of this paper, a long rally is defined as a rally of at least five shots, and a short rally is any rally of length under five.

\begin{table}[ht]
\caption{Feature Summary} % title of Table
\centering % used for centering table
\setlength{\tabcolsep}{3pt}
\scalebox{1.2}{%
\begin{tabular}{c c} % centered columns (4 columns)
\hline\hline %inserts double horizontal lines
Feature & Explanation \\ [0.5ex] % inserts table

%heading
\hline % inserts single horizontal line
SP & percentage of total points won on serve \\
RP & percentage of total points won on receive \\
LRP & percentage of total points won on a long rally \\
SRP & percentage of total points won on a short rally \\
FHP & percentage of total points won on a forehand \\
BHP & percentage of total points won on a backhand \\ 
RANK & player ranking \\
RANKDIFF* & difference in rank between opponents \\
SA* & player serve advantage \\
SRA* & player short rally advantage \\
FHA* & player forehand advantage \\
BALANCE* & measure of how well rounded a player is \\
[1ex] % [1ex] adds vertical space

\hline %inserts single line
\end{tabular}}
%\vspace{4em}

\label{table:nonlin} % is used to refer this table in the text
\end{table}


\subsubsection{Serve Percentage}\label{sp}
The proportion of points won on serve by a player. This can be demonstrated from data similar to Figure \ref{sequence}; if the serve and error are made on opposing sides of the table, it can be concluded that the respective rally was won by the player who served.

\subsubsection{Receive Percentage}\label{rp}
The proportion of points won on receive by a player. Similarly, this can be demonstrated from Figure \ref{sequence}; the serve and error is made by the same player, therefore the point is won by the receiver.

\subsubsection{Long Rally Percentage}\label{lrp}
The proportion of points won on a long rally by a player, determined by the location of the last bounce in a rally. For example, Figure \ref{svlr} indicates that one player won 47 points on a long rally, while the other won 45.

\subsubsection{Short Rally Percentage}\label{srp}
The proportion of points won on a short rally by a player. Similar to \ref{lrp}, Figure \ref{svlr} indicates that one player won 21 points on a short rally, while the other won 24 in the entire match.

\subsubsection{Forehand Percentage}\label{fp}
The proportion of points won on a forehand by a player, determined by the type of stroke used on the winning shot of a rally. For example, Figure \ref{fvbh} indicates one player won 24 points on a forehand, and the other winning 38.

\subsubsection{Backhand Percentage}\label{bp}
The proportion of points won on a backhand by a player. Similar to \ref{fp}, Figure \ref{fvbh} indicates that one player won 43 points on a backhand, and the other winning 29.

\subsubsection{Rank} \label{rank}
Information on both players are included in the dataset, including the ranking of both players.

\subsubsection{Rank Difference} \label{rankdiff}
The feature \textit{RANKDIFF} was constructed by calculating the difference between rankings of two opponents.
$$
RANKDIFF = \begin{cases}
RANK_i - RANK_j &\text{for player $i$} \\
RANK_j - RANK_i &\text{for player $j$} \\
\end{cases}
$$
where $RANK_i$ and $RANK_j$ are the rankings of players $i$ and $j$ respectively at the time of the match. Therefore if a player's rank is better (i.e. lower numerical value) than their opponent's rank, $RANKDIFF$ will be a negative value, and vice versa. However, for some match instances where the ranking of both players are above 100, the feature $RANKDIFF$ is considered to be 0. This is due to the fact that the lower the rank of a player, the more likely it is that there will be other players of a similar standard where rank doesn't accurately represent the standard of a player.

For example, players of rank 2 and rank 7 is much more likely to have an accurate depiction of their standard in comparison to two players of rank 150 and 155, despite the rank difference being the same. Therefore, if both players have a rank of below 100, the expected difference in skill level is considered to have no benefit.

\subsubsection{Serve Advantage} \label{advantage}
The serve advantage of a player is calculated as the difference between their serve and receive winning percentage. This depicts the contrast as to how likely a player is to win a point if they are serving, compared to if they are on receive. Subsequently, the advantage a respective player has in a short rally over a long rally, as well as the advantage a respective player has in a forehand stroke over a backhand stroke, can be calculated therefrom.

\subsubsection{Balance} \label{balance}
An attempt to measure the \textit{completeness} of a player can be calculated by taking the average of serve, short rally and forehand advantage:
$$
BALANCE = \frac{|SA|+|SRA|+|FHA|}{3}
$$
Players of a higher skill level tend to have fewer weaknesses and are stronger in more aspects of the game, therefore the feature $BALANCE$ indicates the overall well-roundness of a player's ability. \label{engineer}

\subsection{Feature Scaling}
Different features tend to have a varying range of values, therefore it is best practice to scale features as part of data pre-processing prior to learning. \textit{Standardization} is a scaling technique to centre values around the mean with a unit standard deviation \cite{bollegala2017dynamic}. We get a coded value by subtracting the mean of the sample from the data and dividing it by the standard deviation. The feature representing rank difference for example, is represented across a much larger range compared to features which are percentages.
